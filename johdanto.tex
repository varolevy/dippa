%Johdanto

Every minute there are 100 hours of video uploaded just to Youtube.com.
~\cite{youtube} 

Analysis, subjective or objective, for video is just starting. In year xx
there where xx articles related to video in IEEE explorer but after two year the amount of
arcticles has skyrockete and there is no end. 

The goal for this thesis is to make state of art analysis of current methods
of video testing. I will not study transmission erros over networks. I focus 
on erros coming from coding, device, optics,
ois(optical image stabilization), etc. I try to focus more on non-reference
methods but reference video methods are also studied and introduced. 

Video quality can be determined with psychophysical experiments, but they are
expensive and time consuming to arrange. There are no physical measure like
meter for distance to predict the quality of video. 

On first chapter I will define what is mobile device. What restrictions it
causes in testing and typical ways of mobile testing.  

Second chapter will introduce dirrerent error types divided into algorithmic
and harware related errors. This will help us to understand how to recognize
those errors and also how to detect those in testing and how to correct them.  

Third chapter digs in to the state of the art of testing methods: 
subjective, objective, reference, non-reference, black box and etc. 

Fourth chapter shows comparision of different methods or practical example
done with matlab. 

Fifth chapter is for conclusions and future work. 
