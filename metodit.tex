%Metodit

\section{State of art in Video testing}

Most widely used methods are PSNR and MSE ~\cite{4347144}
Benefits
*Easy to calculate, easy to compare
Minuses
*Correlation with visible erros low sometimes



\subsection{Reference testing}

\subsection{Non-Reference testing}




TODO: find more watermark based studies and compare them ie: ``A no-reference
vodeo quality assesment method based on digital watermarking''
Model where watermark is added to video and the idea is that after the video
os gone trough the modification pipeline(algorithms, compressions etc) that
once the watermark is extracted from result video the degeneration is about
the same that it would be for the actual video. ~\cite{1203346}



\subsection{Objective testing}

~\cite{4347144}

\subsection{Subjective testing}
http://www.its.bldrdoc.gov/resources/video-quality-research/standards/objective-models.aspx


This is the main thingy also other can be used for ie. 
~\cite{ITU_P910}

Human visual system
*Spatial response
*Temporal response
*Masking
~\cite{4347144}

The history of video quality model validation ~\cite{6659332}
