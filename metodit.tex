%Metodit

\section{State of art in Video testing}

Most widely used methods are PSNR and MSE ~\cite{4347144}
Benefits
*Easy to calculate, easy to compare
Minuses
*Correlation with visible erros low sometimes



\subsection{Reference testing}

\subsection{Non-Reference testing}




TODO: find more watermark based studies and compare them ie: ``A no-reference
vodeo quality assesment method based on digital watermarking''
Model where watermark is added to video and the idea is that after the video
os gone trough the modification pipeline(algorithms, compressions etc) that
once the watermark is extracted from result video the degeneration is about
the same that it would be for the actual video. ~\cite{1203346}



\subsection{Objective testing}

Definition for objective testing is {\ldots} 

At least three different methods are mentioned on many articles and
studies all along. PSNR, SSIM and VQM create to cornerstone for objective
testing. ~cite{4800123}. They all try to evalute quality of saved/received
video, but with differents means in mathematical complexity and therefore the
results also vary when compared to Subjective methods

PSNR and MSE
-is the most commomnly used at least in the literature. 
-Why not in real solutions 
-Easy to calculate, remember five formulas and some example in calulation time
-Correlation with subj. methods quite ppor
SSIM
-At least available to MatLAb quite complex
-Complex to calculate
-Fairly good correlation with sub. methods
VQM
-The harderst to calculate
-Not available
-Why not?
-Good correlation with subj. methods. 

Comparsion/conclusion of objective methods}

ftp://ftp.cs.wpi.edu/pub/techreports/pdf/06-02.pdf


~\cite{4347144}

\subsection{Subjective testing}
http://www.its.bldrdoc.gov/resources/video-quality-research/standards/objective-models.aspx


This is the main thingy also other can be used for ie. 
~\cite{ITU_P910}

Human visual system
*Spatial response
*Temporal response
*Masking
~\cite{4347144}

The history of video quality model validation ~\cite{6659332}
